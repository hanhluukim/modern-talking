\section{Introduction}\label{introduction}

Arguments influence our decisions in many places of our daily life~\cite{Bar-HaimEFKLS2020}.
But with the increasingly larger amount of information found on the Web\footnote{\url{https://internetlivestats.com/}} 
and more effective argument mining, people often need to summarize arguments~\cite{LawrenceR2019,Bar-HaimEFKLS2020}.
\citet{Bar-HaimEFKLS2020} see matching key points to arguments as an intermediate step towards automatically generating 
argumentative summaries~(Section~\ref{related-work}).
The ArgMining~2021 shared task on Quantitative Summarization and Key Point Analysis is the first task on key point matching, 
which is an important step towards summarizing arguments.
Different approaches of matching argument key point pairs, here called matchers, should be proposed and discussed.%
\footnote{\url{https://2021.argmining.org/shared_task_ibm.html}}
%\todo{Cite the task overview paper once the citation is announced.}
For evaluating different argument key point matchers, the shared task organizers introduce a customized mean average 
precision evaluation metric and publish the \ArgKP benchmark dataset~(Section~\ref{data}) to compare 
matchers~\cite{Bar-HaimEFKLS2020}. % \todo{Overview paper}

Pretrained language models like \Bert and \Roberta are nowadays becoming standard approaches to tackle various Natural 
Language Processing tasks~\cite{DevlinCLT2019,LiuOGDJCLLZS2019}.
Because of their extensive pretraining, often fine-tuning these language models with even a small task-specific dataset 
can achieve state-of-the-art performance~\cite{DevlinCLT2019}.
As the \ArgKP dataset~\cite{Bar-HaimEFKLS2020} used in the ArgMining~2021 shared task on Quantitative Summarization is 
relatively small~(24\,083 labelled pairs), we decide to fine-tune \Bert and \Roberta language models rather than train 
a neural classifier from scratch~(Section~\ref{approach}).

Contrasting this neural approach, we introduce a simple rule-based baseline matcher that compares preprocessed tokens of 
each argument to the tokens of each key point~(Section~\ref{approach}).
For the baseline, we compute token overlap after removing stop words, adding synonyms and antonyms, and stemming the 
tokens from both argument and key point using the NLTK~toolkit~\cite{BirdL2004}.

Our fine-tuned \RobertaBase matcher achieves a mean average precision score of up to~0.967 and ranks second in the 
shared task's leaderboard~(Section~\ref{results}).
In a manual error analysis, we find that the imbalanced \ArgKP dataset causes neural models to predict non-matching 
argument key point pairs more precisely than matching pairs~(Section~\ref{error-analysis}).
We further observe a tendency that large length differences between arguments and key points can cause errors.
To encourage researchers to train more robust argument key point matchers, we release our source code under a free license.%
% \footnote{\url{https://github.com/heinrichreimer/modern-talking}}