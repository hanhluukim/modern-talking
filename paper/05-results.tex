\section{Results}\label{results}
\begin{table*}
  \centering
  \caption{Performance of the random and token overlap baseline, \BertBase, and \RobertaBase models with respect to mean average precision~(mAP), precision, and recall of the match label. We report scores for the training, validation, and test set in the strict~(S) and relaxed~(R) label settings, as well as the averages of the two settings~(\(\varnothing\)). The best result per set is highlighted \textbf{bold}.}
  \label{table-results}
  \smaller
  \setlength{\tabcolsep}{2.5mm}
  \begin{tabularx}{\linewidth}{X>{\hspace{1.0em}}ccc<{\hspace{1.0em}}>{\hspace{1.0em}}ccc<{\hspace{1.0em}}>{\hspace{1.0em}}ccc}
    \toprule
    \textbf{Approach} & 
    \multicolumn{3}{c}{\textbf{mAP}} & 
    \multicolumn{3}{c}{\textbf{Precision}} & 
    \multicolumn{3}{c}{\textbf{Recall}} \\
    \cmidrule(l{1em}r{1em}){2-4} \cmidrule(l{1em}r{1em}){5-7} \cmidrule(l{1em}){8-10}
    & S & R & \(\varnothing\) & 
    S & R & \(\varnothing\) & 
    S & R & \(\varnothing\) \\
    \midrule
    \multicolumn{10}{X}{\textit{Training set}} \\
    \midrule
    Random & 
    0.260 & 0.409 & 0.335 & 
    0.173 & 0.330 & 0.252 & 
    0.500 & 0.501 & 0.501 \\
    Token Overlap & 
    0.541 & 0.653 & 0.597 & 
    0.269 & 0.435 & 0.352 & 
    0.323 & 0.275 & 0.299 \\
    \BertBase & 
    0.889 & \textbf{0.981} & 0.935 & 
    \textbf{0.703} & \textbf{0.936} & \textbf{0.819} & 
    \textbf{0.864} & \textbf{0.607} & \textbf{0.736} \\
    \RobertaBase & 
    \textbf{0.915} & 0.979 & \textbf{0.947} & 
    0.702 & 0.927 & 0.814 & 
    0.820 & 0.572 & 0.696 \\
    \midrule
    \multicolumn{10}{X}{\textit{Validation set}} \\
    \midrule
    Random & 
    0.232 & 0.430 & 0.331 & 
    0.180 & 0.364 & 0.272 & 
    0.523 & 0.524 & 0.524 \\
    Token Overlap & 
    0.643 & 0.802 & 0.722 & 
    0.219 & 0.416 & 0.317 & 
    0.390 & 0.366 & 0.378 \\
    \BertBase & 
    0.717 & 0.928 & 0.822 & 
    0.397 & 0.648 & 0.522 & 
    \textbf{0.802} & \textbf{0.649} & \textbf{0.725} \\
    \RobertaBase & 
    \textbf{0.879} & \textbf{0.984} & \textbf{0.932} & 
    \textbf{0.567} & \textbf{0.816} & \textbf{0.692} & 
    0.799 & 0.569 & 0.684 \\
    \midrule
    \multicolumn{10}{X}{\textit{Test set}} \\
    \midrule
    Random & 
    0.237 & 0.355 & 0.296 & 
    0.150 & 0.286 & 0.218 & 
    0.545 & 0.549 & 0.547 \\
    Token Overlap & 
    0.483 & 0.575 & 0.529 & 
    0.232 & 0.350 & 0.291 & 
    0.225 & 0.178 & 0.201 \\
    \BertBase & 
    0.827 & 0.940 & 0.883 & 
    0.326 & 0.526 & 0.426 & 
    \textbf{0.848} & \textbf{0.721} & \textbf{0.784} \\
    \RobertaBase & 
    \textbf{0.913} & \textbf{0.967} & \textbf{0.940} & 
    \textbf{0.490} & \textbf{0.716} & \textbf{0.603} & 
    0.741 & 0.569 & 0.655 \\
    \bottomrule
  \end{tabularx}
\end{table*}


Submissions to the ArgMining~2021 shared task on Quantitative Summarization and Key Point Analysis are evaluated with 
respect to mean average precision~\cite{kpa-2021-overview}. 
%\footnote{\url{https://2021.argmining.org/shared_task_ibm.html}}
%\todo{Overview paper.}
The organizers calculate the score by pairing each argument with the best matching key point according to the predicted 
matching probabilities. 
Within each topic-stance combination, only 50\,\% of the arguments with the highest predicted matching score are then 
considered for evaluation. 
The task organizers claim that this removal of 50\,\% of the pairs is necessary because some arguments do not match any 
of the key points, which would influence mean average precision negatively~\cite{kpa-2021-overview}. 
For the remaining argument key point pairs in each topic-stance combination, the average precision is calculated and the 
final score is computed as the mean of all average precision scores.

The task organizers consider two evaluation settings: strict and relaxed.
Both settings are based on the ground-truth labels from the \ArgKP dataset~\cite{Bar-HaimEFKLS2020}. The two evaluation settings are created to account for argument key point pairs in the \ArgKP with undecided labels~(i.e. not enough agreement between annotators). In the strict setting,
the shared task organizers consider those undecided pairs as \texttt{no-match}. In the relaxed setting, however, the shared task organizers consider the undecided pairs as \texttt{match}~\cite{kpa-2021-overview}. % \todo{Overview paper.}
The mean average precision score is then calculated in the two settings based on the ground-truth labels and the derived labels for the undecided pairs. 
We stress that in this complex evaluation setup, the mean average precision score in the relaxed setting would 
favor assuming matches in case of model uncertainty. 
In comparison, in the strict setting mean average precision would favor assuming \texttt{no-match} between an argument and key point.
However, we find that because only the most probable matching key point is being considered for evaluation, this effect is minor.
The evaluation score in general favors matchers that can match a single key point for each argument with high precision.
It is however not important if a matcher does predict non-matches with high certainty.

\subsection{Discussion}
In Table~\ref{table-results}, we report mean average precision in the strict and relaxed settings of the training, 
validation, and test set in the \ArgKP dataset.
We complement the mean average precision scores by adding precision and recall scores of the \texttt{match} label, 
both in the strict and relaxed setting. To calculate precision and recall, we label an argument and key point pair as \texttt{match} if their score is higher than~0.5 and as \texttt{no-match} otherwise.
To aggregate results of the strict and relaxed settings, we also report the average score of the two variants.
The reported scores should allow for automated and unbiased evaluation of our models and easier comparison with competitive approaches.
We report all 27~scores for the token overlap baseline model as well as for the fine-tuned \BertBase and \RobertaBase models.
To make our results more comparable, we add a second baseline, where matches between arguments and key points of same 
topic and stance are predicted with uniform random probability.
That random baseline represents a worst-case matcher and any weak matcher should exceed its evaluation scores.

The token overlap baseline achieves a mean average precision of~0.483 in the strict setting and~0.575 in the relaxed 
setting on the test set.
Thus, it is nearly twice as good as a random matcher with respect to mean average precision.
Even though this baseline has reasonably good scores on all datasets, we are concerned about the large discrepancies 
between its scores on the validation set and the training and test dataset. 
The rather simple baseline captures the similarity between an argument and a key point on the token level and might be sensative against more complicated parphrases.

Both fine-tuned matchers outperform the baselines by a large margin.
While the \BertBase matcher achieves higher relaxed mean average precision on the training set than the \RobertaBase 
matcher, the \RobertaBase matcher is overall better than \Bert, especially with strict labels.
The \RobertaBase matcher achieves a mean average precision of~0.913 in the strict setting on the test set, and~0.967 
in the relaxed setting.
As these scores on the test set are nearly as high as on the training set, we argue that \Roberta is a more robust 
language model and generalizes better than \Bert.
Also the \RobertaBase matcher performs better in terms of precision, while the \BertBase matcher is better with respect to 
recall for all dataset splits in both the strict and relaxed settings.
