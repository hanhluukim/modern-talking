\section{Related Work} 

    \subsection{Key Point Analysis}
        Given the first track of the shared task analyzing key points is crucial. In their work \cite{Bar-HaimEFKLS2020} 
        Bar-Haim et al. propose an approach for summarising large argument collections to small sets of key points. Thus
        covering a sufficient amount of all arguments. They show that domain experts can very quickly
        create pro and con key points, which are able to "capture the gist" of the arguments on the given topic. All this
        without being exposed to the arguments themself. Furthermore they develop the large-scale dataset ArgKP
        which is the foundation of this shared task. 
        In a later work \cite{Bar-HaimKEFLS2020} Bar-Haim et al. construct an automatic method for key point 
        extraction which can compete with key points created by human domain experts. The method consists of two aspects. 
        Assuming that the key points can be found among the given comments 
        they first select short, high quality comments as key point candidates an then select the candidate with the highest
        data coverage. Using the HuggingFace transformer framework they fine-tune four different models from which 
        ALBERT \cite{lan2019albert} has the best F1 score with $0.809$ but RoBERTa \cite{liu2019roberta} (F1 score of 
        $0.773$) is chosen for key point extraction since it has a 6 times faster inference time. 
        In the paper \cite{egan2016summarising}, Egan et al. propose a summarising for informal arguments such as they
        occure in online political debates. By extracting verbs and their syntactic arguments they retrieve points which
        can make key content accessible. By grouping these points they propose to create discussion summaries.

    \subsection{Argument Clustering}
        Argument Clustering is a mighty tool that enables algorithms to assign multiple arguments, which adress a similar
        key message to a given topic. In the paper \cite{reimers2019classification}, Reimers et al. make use of 
        "contextualize word embeddings to classify and cluster topic-dependet arguments". Having performed argument
        classification they then compute similar and dissimilar pairs of arguments. Two approaches one with clustering
        and one without are being used. 
        Clustering arguments is achieved by usage of agglomerative hierarchical clustering \cite{day1984efficient}. 
        Without clustering a fine-tuned BERT-base-uncased model reached a F1 mean score for similar and dissimilar
        arguments of $0.7401$. 
        Agglomerative hierarchical clustering being a strict partitioning algorithm, results for clustering perform
        worse by up to 7.64pp (Bert-large F1 mean score: $0.7135$). Hence they conclude that "strict partitioning 
        clustering methods introduce a new source of errors".

        

    \subsection{Stance Classification}
