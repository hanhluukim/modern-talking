\section{Related Work}\label{related-work}

% \todo{Shorten subsections. We probably don't need subsection headers in this section.}

% \subsection{Key Point Analysis}



Similar tasks to key point analysis include clustering arguments~\cite{reimers2019classification,ajjour2019modeling}, detecting similar arguments in a pairwise fashion~\cite{misra:2016} and matching arguments to generic-arguments~\cite{naderi:2017}. Using points to summarize arguments were approached by~\citet{egan2016summarising} on online discussion. Points were extracted by using the verbs and their syntactic arguments and are then clustered together to deliver a summary of the discussion. 

Key point analysis is the task of matching a given argument with one or more pre-defined key points~\cite{Bar-HaimEFKLS2020}. To develop models for the task, \citet{Bar-HaimEFKLS2020} introduced a dataset (\ArgKP) which contains 24 093~argument key point pairs on 28~topics. Each argument and key point is labeled manually as \texttt{match} or \texttt{no-match}. The authors experimented with several unsupervised and supervised approaches to perform the task in a cross-topic experimental setting. BERT~\cite{DevlinCLT2019} performed the best in their experiments by reaching an F1 score of $0.68$.

In a later work, \citet{Bar-HaimKEFLS2020} develop a summarization approach for online discussions that
uses key point analysis. The summrization approach takes as input a set of comments on a given topic and returns a set of representative key points from them, each key point with the count of matched comments. In its essence, the summarization approach uses a matching model that gives a score for a given comment and key point or a couple of key points. For matching models, \citet{Bar-HaimKEFLS2020} compare different variants of BERT~\cite{DevlinCLT2019}. Among the tested models, ALBERT \cite{lan2019albert} performed the best with an F1 score $0.809$, but RoBERTa~\cite{LiuOGDJCLLZS2019} were chosen for key point extraction at the end, which is 6 times faster than ALBERT and still achieves an F1 score of $0.773$. 

Our approaches for the key point analysis are based on \Bert and \Roberta. \Bert stands for Bidirectional Encoder Representations from Transformers and is an open-source bidirectional language representation model published by Google~\cite{DevlinCLT2019}. \Bert is based on the transformer architecture; however, uses the encoder in multi-layers. 
\Bert is pre-trained over unlabeled text to learn a language representation and can be fine-tuned on downstream tasks. During pre-training, \Bert is trained on two unsupervised tasks: Masked Language Model and Next Structure Prediction. \Roberta is an improved variant of \Bert that is introduced by Facebook in 2019~\cite{LiuOGDJCLLZS2019}. \citet{LiuOGDJCLLZS2019} tweaked \Bert by using a larger training data size of 160GB of uncompressed text, more compute power, larger batch-training size, and optimized hyper-parameters. In comparison to \Bert, pre-training tasks for \Roberta were done with full-length sentences and include only Masked Language Model while applying different masks in each training epoch (dynamic masking). \Roberta outperforms BERT on all 9 GLUE tasks in the single-task setting and 4 out of 9 tasks in the ensembles setting~\cite{WangSMHLB2018,LiuOGDJCLLZS2019}.


% \subsection{Argument Clustering}
% Argument Clustering is a mighty tool that enables algorithms to assign multiple arguments, which adress a similar
% key message to a given topic. \citet{reimers2019classification} make use of 
% "contextualize word embeddings to classify and cluster topic-dependet arguments". Having performed argument
% classification they then compute similar and dissimilar pairs of arguments. Two approaches one with clustering
% and one without are being used. 
% Clustering arguments is achieved by usage of agglomerative hierarchical clustering \cite{day1984efficient}. 
% Without clustering a fine-tuned BERT-base-uncased model reached a F1 mean score for similar and dissimilar
% arguments of $0.7401$. 
% Agglomerative hierarchical clustering being a strict partitioning algorithm, results for clustering perform
% worse by up to 7.64pp (Bert-large F1 mean score: $0.7135$). Hence they conclude that "strict partitioning 
% clustering methods introduce a new source of errors".
% Another approach proposed by \citet{ajjour2019modeling} revolves around clustering 
% arguments into so called frames which are "a set of arguments that focus on the same aspect". 
% Thereby framing \cite{entman1993framing} only a specific information to present to the listeners and convince 
% them of your stance.
% They propose that an argument consists of two crucial parts. The topic and the frame.
% Hence their approch splits into three steps: First, all arguments are clustered into $m$ topics. Second,
% topical features are extracted from all arguments and therefore from its cluster. Third, the arguments are 
% reclustered into $k$ non-overlapping frames. By utilizing k-means \cite{hartigan1979ak} for clustering
% and Term Frequency-Inverse Document Frequency (TF-IDF) for topic removal they achieved a F1 score of 
% $0.28$.

% \subsection{Pretrained Language Models}



