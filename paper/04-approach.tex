\section{Approach}

To match key points to arguments, we propose two different approaches.
First, we shortly discuss a simple yet effective baseline measuring term overlap between key points and arguments.
Second, to improve upon the simple baseline, we introduce a fine-tuning approach using \Bert~\todocite and \Roberta~\todocite pretrained language models. We use both language models in standard configuration with only minor changes highlighted below.

\subsection{Baseline}

We start with a very simple baseline. Therefore choosing a Term Overlap baseline with preprocessed terms. 
Generally it can be assumed, that key-points are summarizing ideas of all associated arguments. We therefore came up with the idea
that certain key-words, contained in a lot of arguments, are also very likely to be present in the associated key-point. This makes 
sense with our intuition, because rather than using completly new words for summarization of arguments, a human would 
rather reuse certain important words, which have been already found in the arguments.\\
\todo{Maybe this example fits better into the data section?}
For example, the following argument exists in the \ArgKP dataset:\\
\textit{People reach their limit when it comes to 
their quality of life and should be able to end their {\color{blue} suffering}. This can be done with little 
or no {\color{blue} suffering} by {\color{orange} assistance} and the person is able to say good bye.}\\ 

In relation to this the following key-point can be found:\\
\textit{{\color{orange} Assisted} suicide reduces {\color{blue} suffering}.}\\

It can already be seen, that an overlap with the words \textit{suffering} exists. 
We can further increse this overlap by performing simple preprocessing steps.\\
First of all, we utilize stop word removal for reducing the noise within all arguments. Initially this can be seen 
counter productive, because less words means less overlap and therefore worse performance. But at second glance this 
makes a lot of sense. A lot of arguments and key-points contain unnecessary words like \textit{the, and, as etc.}.
Removing these gives us purer sentences and results in less confusion with the term-overlap algorithm. Furthermore the 
redundancy of language makes it possible to contain key-aspects in sentences, even with out these unnecessary stop words.\\
Secondly, Stemming reduces terms to their corresponding stems and thus achieves a better generalization, 
when comparing terms. For example, the word: \textit{weakness} will be stemmed to \textit{weak} using the Porter-Stemmer 
\cite{Porter1980}. Thus creating an overlap between those words and increasing the possibility that an argmunt containing
\textit{weakness} will be associated to a key-pont containing \textit{weak}.\\
Thirdly, we increase the generalization of our term-overlap algorithm even further by creating lists of synonyms and 
antonyms and testing if checked words can be replaced with candidates from these to increase overlap.\\
For the actual similarity computation of given arguments and key-points we use the Jaccard similarity coefficient 
\cite{Jaccard1902}. Meaning a higher proportion of terms that appear in an argument as well as in a key-point will 
classify this key-point to be more likely to match.

\subsection{Language Model Fine-tuning}

